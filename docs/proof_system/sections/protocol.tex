\section{RedShift Protocol}
\label{section:protocol}
\textbf{WIP}

Notations:
\begin{itemize}
	\item $N_{\texttt{wires}}$ is the number of wires (`advice columns`).
	\item $N_{\texttt{perm}}$ is the number of wires that are included in the permutation argument
	\item $N_{\texttt{sel}}$ is the number of selectors used in the circuit
	\item $N_{\texttt{const}}$ is the number of constant columns 
	\item Permutation over the table: $\sigma(\text{column : } i, \text{row : } j) = (\text{column : } i', \text{row : } j')$
	\item $\textbf{f}_i$ is witness polynomial for $0 \leq i < N_{\texttt{wires}}$
	\item $\textbf{f}_{c_i}$ is constant-related polynomial for $0 \leq i < N_{\texttt{wires}}$
	\item $\textbf{gate}_i$ is gate polynomial for $0 \leq i < N_{\texttt{const}}$
\end{itemize}

 $\texttt{adv}$ be the number of advice columns, $\tau$ be the number the fixed columns. 
Let . 

For details on polynomial commitment scheme and polynomial evaluation scheme, we refer the reader to \cite{cryptoeprint:2019:1400}.

Preprocessing:
\begin{enumerate}
	\item $\mathcal{L}' = (\textbf{q}_{0}, ..., \textbf{q}_{N_{\texttt{sel}}})$
	\item Let $\omega$ be a $2^k$ root of unity
	\item Let $\delta$ be a $T$ root of unity, where $T \cdot 2^S + 1 = p$ with $T$ odd and $k \leq S$
	\item Compute $N_{\texttt{perm}}$ permutation polynomials $S_{\sigma_i}(X)$ such that $S_{\sigma_i}(\omega^j) = \delta^{i'} \cdot \omega^{j'}$
	\item Compute $N_{\texttt{perm}}$ identity permutation polynomials: $S_{id_i}(X)$ such that $S_{id_i}(\omega^j) = \delta^i \cdot \omega^j$
\end{enumerate}

Protocol:
\begin{enumerate}
	\item \textbf{P}:
	\begin{enumerate}
		\item Choose masking polynomials: 
		\begin{center}
			$h_i(x) \leftarrow \mathbb{F}_{<k}[x]$ for $0 \leq i < N_{\texttt{wires}}$
		\end{center}
		\item Define new witness polynomials:
		\begin{center}
			$f_i(x) = \textbf{f}_{i}(x) + h_i(x)Z(x)$ for $0 \leq i < N_{\texttt{wires}}$
		\end{center}
	\end{enumerate}
	\item \textbf{V}:
	\begin{enumerate}
		\item Send to \textbf{P}: $\beta, \gamma \leftarrow \mathbb{F}$
	\end{enumerate}
	\item \textbf{P}:
	\begin{enumerate}
		\item Compute for $0 \leq j < N_{\texttt{perm}}$
		\begin{center}
			$p_j = f_j + \beta \cdot S_{id_j} + \gamma$ \\
			$q_j = f_j + \beta \cdot S_{\sigma_j} + \gamma$
		\end{center}
		\item Define:
		\begin{center}
			$p'(X) = \prod\limits_{0 \leq j < N_{\texttt{perm}}} p_j(X) \in \mathbb{F}_{<N_{\texttt{perm}} \cdot n}[X]$ \\
			$q'(X) = \prod\limits_{0 \leq j < N_{\texttt{perm}}} q_j(X) \in \mathbb{F}_{<N_{\texttt{perm}} \cdot n}[X]$
		\end{center}
		\item Compute $P(X), Q(X) \in \mathbb{F}_{<n+1}[X]$, such that:
		\begin{center}
			$P(g) = Q(g) = 1$ \\
			$P(g^i) = \prod\limits_{1 \leq j < i}p'(g^i)$ for $i \in {2, \dots, n + 1}$ \\
			$Q(g^i) = \prod\limits_{1 \leq j < i}q'(g^i)$ for $i \in {2, \dots, n + 1}$ \\
		\end{center}
		\item Compute and send commitments to $P$ and $Q$ to \textbf{V}
	\end{enumerate}
	\item \textbf{V}:
	\begin{enumerate}
		\item Send to \textbf{P}: $a_1, \dots, a_6 \leftarrow \mathbb{F}$
	\end{enumerate}
	\item \textbf{P}:
	\begin{enumerate}
		\item Define polynomials ($F_1, \dots, F_5$ - copy-satisfability):
		\begin{center}
			$F_1(x) = L_1(x)(P(x) - 1)$\\
			$F_2(x) = L_1(x)(Q(x) - 1)$ \\
			$F_3(x) = P(x)p'(x) - P(xg)$ \\
			$F_4(x) = Q(x)q'(x) - Q(xg)$ \\
			$F_5(x) = L_n(x)(P(xg) - Q(xg))$ \\
			$F_6(x) = \sum\limits_{0 \leq i < N_{\texttt{sel}}} (\textbf{q}_{i}(x) \cdot \texttt{gate}_i(x))
				+ (\sum\limits_{0 \leq i < N_{\texttt{const}}}(\textbf{f}_{c_i}(x)) + PI(x))$
		\end{center}
		\item Compute:
		\begin{center}
			$F(x) = \sum\limits_{i = 1}^6 a_iF_i(x)$ \\
			$T(x) = \frac{F(x)}{Z(x)}$
		\end{center}
		\item Split $T(x)$ into seprate polynomials $T_0(x), ..., T_{N_{\texttt{perm}} + 1}$
		\item Send commitment to $T_0(x), ..., T_{N_{\texttt{perm}} + 1}$ to \textbf{V}
	\end{enumerate}
	\item \textbf{V}:
	\begin{enumerate}
		\item Send to \textbf{P}: $y \leftarrow \mathbb{F}/H$
	\end{enumerate}
	\item \textbf{P}:
		\item Run evaluation scheme over committed polynomials and $y$
	\item \textbf{V}
	\begin{enumerate}
		\item Checks the identity:
		\begin{center}
			$\sum\limits_{i = 1}^6a_iF_i(y) = Z(y)T(y)$
		\end{center}
	\end{enumerate}
\end{enumerate}