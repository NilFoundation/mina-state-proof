\section{Introduction}
\label{section:introduction}

To prove Mina blockchain's state on the Ethereum Virtual Machine, we use Redshift SNARK\cite{cryptoeprint:2019:1400}.
RedShift is a transparent SNARK that uses PLONK\cite{cryptoeprint:2019:953} proof system but replaces the commitment scheme.
The authors utilize FRI\cite{ben2018fast} protocol to obtain transparency for the PLONK system.

However, FRI cannot be straightforwardly used with the PLONK system.
To achieve the required security level without huge overheads, the authors introduce \textit{list polynomial commitment} scheme as a part of the protocol.
For more details, we refer the reader to \cite{cryptoeprint:2019:1400}.

The original RedShift protocol utilizes the classic PLONK\cite{cryptoeprint:2019:953} system.
To provide better performance, we generilize the original protocol for use with PLONK with custom gates \cite{turbo}, \cite{plonkhalo2} and lookup arguments \cite{cryptoeprint:2020:315}, \cite{lookuphalo2}.\footnote{Details on the proof system: \url{https://github.com/NilFoundation/evm-mina-verification/tree/master/docs/proof_system}}